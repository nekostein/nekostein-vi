\documentclass[a4paper,11pt]{report}
% Ver 2023-08-27T18:52:42+00:00
\usepackage[textwidth=38em]{geometry}
\usepackage{calc,fontspec,tocloft,longtable,tabu,booktabs,xcolor,datetime2,paralist,enumitem,hyphenat,graphicx,eso-pic}
\usepackage{amssymb,amsmath,multicol}
\usepackage[hidelinks]{hyperref}
\usepackage{xeCJK,xeCJKfntef}
\xeCJKsetup{PunctStyle=plain,RubberPunctSkip=false,AllowBreakBetweenPuncts=true,LoadFandol=false,WidowPenalty=1000,CJKglue=\strut\hskip 0pt,CJKecglue=\strut\hskip 0.35em plus 5em minus 0.09em}
\setmainfont{CMU Serif}
\setromanfont{CMU Serif}
\setsansfont{Roboto}
\setmonofont{JetBrains Mono NL}
\setCJKmainfont{Noto Serif CJK SC}
\setCJKromanfont{Noto Serif CJK SC}
\setCJKsansfont[BoldFont={Noto Sans CJK SC Bold}]{Noto Sans CJK SC}
\setCJKmonofont[BoldFont={Noto Sans CJK SC Bold}]{Noto Sans CJK SC}

\setdefaultleftmargin{2em}{2em}{2em}{2em}{2em}{2em}

\setlength{\parindent}{0pt}
\setlength{\parskip}{8pt}
\linespread{1.18}
\setlength{\tabulinesep}{3.5pt}
\frenchspacing

\usepackage{sectsty}
\allsectionsfont{\sffamily\raggedright}
\newcommand{\contentsstyle}[0]{\sffamily}
\renewcommand{\contentsname}[0]{\contentsstyle Contents}




\newcommand{\nsreportcover}[5]{
    % This command is maintained in ns-report, but also appears in ns-article
	% argv: date, author, title, department, annotation
	\begin{titlepage}
        \newgeometry{hmargin=36mm,tmargin=45mm,bmargin=36mm}
		\parskip=3pt
        \pagecolor{black!91}
        \color{white}

        \flushright
		\includegraphics[width=50pt]{_dist/libvi/geologo/Nekostein-geologo.white_null.png}\par
		\vskip 140pt

        % Report title is here...
        \flushleft
        \sffamily
        \fontspec[Numbers=Monospaced]{Inter Tight Medium}
		% \Huge
		\fontsize{31pt}{37pt}\selectfont
        \raggedright
		\strut#3

		\vfill
        
		% \Large
        \fontspec[Numbers=Monospaced]{Inter Tight Medium}
        \fontsize{14pt}{18pt}\selectfont
		\strut#2\par
		\strut#1\par
		\vskip 80pt

        \flushleft
		% \normalsize
        \fontsize{10.95pt}{13pt}\selectfont
        \mdseries
		\strut#4\hfill%
		\strut#5\par
	\end{titlepage}
    \restoregeometry
    \pagecolor{white}
    \color{black}
}



\hypersetup{pdftitle={Nekostein VI Manual}}



\newcommand{\altfbox}[1]{%
    \colorbox{black}{#1}%
}



\begin{document}
\nsreportcover{\today}{Neruthes}{Nekostein VI Manual}{Nekostein HQ}{For Public Release}
\sffamily

\tableofcontents\clearpage




\chapter{Preface}
This manual governs the using of brand visual assets of Nekostein.
When you need to use our logo, please follow the instructions included in this document.
Brand integrity is an important matter for us, as it should be for all businesses.

You may download our VI assets by visiting the ``nekostein-vi'' repository on GitHub at the following URL.

\begin{center}
    \href{https://github.com/nekostein/nekostein-vi}{https://github.com/nekostein/nekostein-vi}
\end{center}






\chapter{Geometric Mark}
\altfbox{\includegraphics[height=90pt]{_dist/wwwmisc/avatar/Nekostein-avatar.white_black.png}}~~%
\altfbox{\includegraphics[height=90pt]{_dist/wwwmisc/avatar/Nekostein-avatar.D3E9CA_0E497E.png}} (Good)
\hfill
\altfbox{\includegraphics[height=90pt]{_dist/doc-example/avatar/Nekostein-avatar.black_white.png}} (Bad)

Always use bright body and dark background. Prefer using white body and black background.

You would generally use the provided ``Nekostein-avatar.white\_black.png'' file.
It is great for most scenarios, including square avatar, circle avatar, icon grid, and free canvas.

You may also play with the ``Nekostein-geologo.white\_null.svg'' file in your own vector editor to apply custom effects.

For your convenience, if you need a rounded square,
you may use the provided file ``Nekostein-avatar.white\_black.rounded.png''.





\chapter{Word Mark}
\altfbox{\includegraphics[height=90pt]{_dist/wwwmisc/wordmark/Nekostein-logo-square.white.png}}~~%
\altfbox{\includegraphics[height=90pt]{_dist/wwwmisc/wordmark/Nekostein-logo-square.black.png}}

Our word mark uses the \textbf{Myriad Pro} font (the Black face) with a 10\% letter spacing.
See ImageMagick documentation for explanations on ``-pointsize'' and ``-kerning''.
We use the two parameters (``-pointsize 700 -kerning 70'') to create the word mark PNG.

You would generally use the provided ``Nekostein-logo.white.png'' file.






\chapter{Logo Combinations}

\section{Vertical Combination}
\altfbox{\includegraphics[height=90pt]{_dist/wwwmisc/logocomb/Nekostein-logocombV-padded.white_black.png}}

When a combination is needed, the vertical combination is often a nice choice.
You may use the ``Nekostein-logocombV.white.png'' file over dark background.

Also, the ``Nekostein-logocombV-padded.white\_black.png'' file includes internal padding and black background.
You may use this image over bright background.

\section{Horizontal Combination}
\altfbox{\includegraphics[height=2.85em]{_dist/wwwmisc/logocomb/Nekostein-logocombH.white.png}}\hfill%
\colorbox{white}{\includegraphics[height=2.85em]{_dist/wwwmisc/logocomb/Nekostein-logocombH.black.png}}

The horizontal combination may fit small-size scenarios, e.g. listing sponsors at the bottom of a poster.
You may use the ``Nekostein-logocombH.*.png'' files according to your background color.












\chapter{Guilloche Patterns}
We have several guilloche patterns to be used.
To reduce the size of the distributed ZIP archive, we only include the SVG versions.

You can find them in ``\_dist/libvi/patterns''.

To generate the PNG versions, run the following command.

\begin{verbatim}
./_dist/libvi/distutils/patternpng.sh
\end{verbatim}

% {\includegraphics[width=\linewidth]{_dist/libvi/patterns/p01.small.jpg}}

% {\includegraphics[width=\linewidth]{_dist/libvi/patterns/p02.small.jpg}}

% {\includegraphics[width=\linewidth]{_dist/libvi/patterns/p03.small.jpg}}









\chapter{NekstFair}
\section{NekstFair Logo}
\colorbox{white}{\includegraphics[width=16em]{_dist/wwwmisc/nekstfair/nekstfair.black.png}}%
\hfill%
\colorbox{black}{\includegraphics[width=16em]{_dist/wwwmisc/nekstfair/nekstfair.white.png}}

The logo of NekstFair uses the Rubik font.

The ``nekstfair.*.png'' files may be used in simple black-and-white environments.
Feel free to apply custom colors when necessary.






\chapter{Afterwords}
This manual is published for informational purposes only,
and is not a legal statement permitting any actual use of our brand assets.

We generally welcome good-faith uses like press coverage,
but we reserve the right to disapprove.

If you have any question, please open an issue in the repository on GitHub.
Alternatively, you may contact the maintainer of this VI manual.

Maintainer: Neruthes\\
Email: neruthes@outlook.com




\end{document}
