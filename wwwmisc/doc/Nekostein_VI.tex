\documentclass[a4paper,11pt]{article}
\input{.texlib/std-article.tex}
\usepackage{tcolorbox}

\hypersetup{pdftitle={Nekostein VI Manual}}

\begin{document}
\stdtitle{Nekostein VI Manual}{Neruthes @ Nekostein}{Last revision \today}
\sffamily


\newcommand{\altfbox}[1]{%
    \colorbox{black}{#1}%
}







\section{Abstract}
This manual governs the using of brand visual assets of Nekostein.
When you need to use our logo, please follow the instructions included in this document.
Brand integrity is an important matter for us, as it should be for all businesses.

You may download our VI assets by visiting the ``nekostein-vi'' repository on GitHub at the following URL.

\begin{center}
    \href{https://github.com/nekostein/nekostein-vi}{https://github.com/nekostein/nekostein-vi}
\end{center}






\section{Geometric Mark}
\altfbox{\includegraphics[height=90pt]{_dist/wwwmisc/avatar/Nekostein-avatar.white_black.png}}~~%
\altfbox{\includegraphics[height=90pt]{_dist/wwwmisc/avatar/Nekostein-avatar.D3E9CA_0E497E.png}} (Good)
\hfill
\altfbox{\includegraphics[height=90pt]{_dist/doc-example/avatar/Nekostein-avatar.black_white.png}} (Bad)

Always use bright body and dark background. Prefer using white body and black background.

You would generally use the provided ``Nekostein-avatar.white\_black.png'' file.
It is great for most scenarios, including square avatar, circle avatar, icon grid, and free canvas.

You may also play with the ``Nekostein-geologo.white\_null.svg'' in your own vector editor to apply custom effects.

For your convenience, if you need a rounded square, you may use the provided file ``Nekostein-avatar.white\_black.rounded.png''.





\section{Word Mark}
\altfbox{\includegraphics[height=90pt]{_dist/wwwmisc/wordmark/Nekostein-logo-square.white.png}}~~%
\altfbox{\includegraphics[height=90pt]{_dist/wwwmisc/wordmark/Nekostein-logo-square.black.png}}

Our word mark uses the \textbf{Brygada 1918} font (the Bold face) with a 10\% letter spacing.
See ImageMagick documentation for explanations on ``-pointsize'' and ``-kerning''.
We use the two parameters (``-pointsize 700 -kerning 70'') to create the word mark PNG.

You would generally use the provided ``Nekostein-logo.white.png'' file.






\section{Logo Combinations}

\subsection{Vertical Combination}
\altfbox{\includegraphics[height=90pt]{_dist/wwwmisc/logocomb/Nekostein-logocombV-padded.white_black.png}}

When a combination is needed, the vertical combination is often a nice choice.
You may use the ``Nekostein-logocombV.white.png'' file.

Also, the ``Nekostein-logocombV-padded.white\_black.png'' file includes internal padding and black background.

\subsection{Horizontal Combination}
\altfbox{\includegraphics[height=2.85em]{_dist/wwwmisc/logocomb/Nekostein-logocombH.white.png}}\hfill%
\colorbox{white}{\includegraphics[height=2.85em]{_dist/wwwmisc/logocomb/Nekostein-logocombH.black.png}}

In rare occasions you will want to use the horizontal combination.
You may use the ``Nekostein-logocombH.*.png'' file.








\section*{Afterwords}
This manual is published for informational purposes only,
and is not a legal statement permitting any actual use of our brand assets.

We generally welcome good-faith uses like press coverage,
but we reserve the right to disapprove.

If you have any question, please open an issue in the repository on GitHub.
Alternatively, you may contact the maintainer of this manual at ``neruthes (at) outlook.com''.




\end{document}
